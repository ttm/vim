\documentclass{article}
\usepackage{graphicx}
\usepackage[usenames,dvipsnames]{xcolor}
\usepackage{hyperref}
\hypersetup{
	%pagebackref=true,
	pdfcreator={LaTeX with abnTeX2},
	pdfkeywords={abnt}{latex}{abntex}{USPSC}{trabalho acadêmico}, 
	colorlinks=true,       		% false: boxed links; true: colored links
	linkcolor=blue,          	% color of internal links
	citecolor=blue,        		% color of links to bibliography
	filecolor=magenta,      		% color of file links
	urlcolor=blue,
	allbordercolors=black,
	bookmarksdepth=4
}
\usepackage[utf8]{inputenc}
\newcommand{\ttt}[1] {
	\texttt{<#1>}}
\newcommand{\tttt}[1] {
	\texttt{#1}}

\begin{document}
\title{An anthropological account of the Vim editor:\\
ouroboros, features and tweaks after 10 years of usage}
% Anthropological computation
\author{Renato Fabbri\\
\texttt{renato.fabbri@gmail.com}\\
University of São Paulo,\\
Institute of Mathematical and Computer Sciences\\
São Carlos, SP, Brazil
}
\maketitle
\begin{abstract}
The Vim editor is very rich in capabilities
and thus complex.
This article is a description of the Vim text editor
and a set of considerations on its usage and design.
It is the result of more than ten years of experience
with Vim for writing and editing various types of documents,
e.g.:
Python, C++, JavaScript, ChucK, programs;
and \LaTeX, Markdown, HTML, RDF, Make and other markup files;
binary files.
It is said that it takes about ten years to master (or start mastering)
this text editor, and I find that other experienced users
have a different view of Vim and that they use a different
set of features.
Therefore, this document exposes my insights in order
to confront my usage with that of other Vim users.
Another goal is
to make available a reference document with which new users
can grasp a sound overview by reading it and the discussions that
it might generate.
Also, it should be useful for users of any degree of experience,
including me, as a compendium of commands, namespaces and tweaks.
Upon feedback, and maturing of my Vim usage,
this document might be enhanced or receive additional
material.
\end{abstract}

\section{Introduction}\label{intro}
Vim is a very complex editor,
considered by the Linux community to be
matched only by Emacs.
They both are the standard advanced text editors
of the free software and open source communities
and have been developed for decades.
This document describes the Vim editor
and proposes a set of enhancements of the user
experience through simple tweaks.
The editor has a very mature documentation
and the contents herein presented is a
report on the overall understandings I
have of Vim after a little bit more
of ten years using it.
The purposes of this document are:
\begin{itemize}
  \item to help new users in grasping Vim essentials
  and convenient practices.
  \item To attain a sound overall description of the editor.
  \item To record the view of the editor that
  a user (me) has after $10$ years of usage.
  \item To confront my usage with that of other experienced
  users. This is helpful for me, but also for the other users
  as they might benefit from this content and from discussions
  that might arise from it.
  \item To propose some enhancements to Vim through simple tweaks and potential plugins.
\end{itemize}

Advanced users might just skim throughj
Section~\ref{basics} and consider more carefully
Section~\ref{namespaces}.
The concluding remarks and proposed enhancements
might also deserve some attention.

I've had experience with other editors, e.g. Kate, gedit, and Notepad2.
I used Vim for writing and editing computer code (Python, Javascript, C++, ChucK, bash, etc), markup languages (HTML, CSS, RDF, Markdown, \LaTeX, etc) and binary files.
Eventually, I edited database files and other types os files.
With Vim, I mostly write software (web and scientific),
music, poems and short stories.
It is very useful because:
\begin{itemize}
  \item it is meant to be a plain text (e.g. ascii, utf8) editor
  and does not (by standard) insert special charaters (e.g. for formating, with binary instructions).
  \item It has a powerful architecture and set of commands.
  \item It is highly configurable and most often the users
  have a set of commands for standard settings and hacks kept in the vimrc and other configuration files.
\end{itemize}

The standard capabilities of the editor
should become clear in Section~\ref{basics}.
Vim is constantly evolving and there are many plugins,
some of them very popular for both general and specific
users.
Accordingly, there are many possible enhancements,
and Section~\ref{conc} report the most prominent of them
for me and potential workarounds made available as plugins.

This document is written is a DRY KISS
(Don't Repeat Yourself, Keep It Simple Stupid) style.
Complex is to master the use of Vim
and one finds sound
references in help files and a nice vimrc.
Therefore the following content should be kept
as uncomplicated and original as possible.
Also, because of Vim's complexity and entailed
bond of this document to my usage,
there is an anthropological component
which is evdent in the use of the first person
in sentences.
This can be understood as anthropological computation
or physics~\cite{anPh,anPh2} and observed to help
in the technological groundwork of the civil society.


% Vi and Vim
\subsection{Historical note}
Vim was first released publicly in 1991.
It is a cross-platform GNU licensed free and open source extended clone of Bill Joy's vi text editor.
Vim's development is coordinated (and performed) since the beginning
mainly by Bram Moolenaar.
Today, current bleeding-edge version is 8.0.1257.
I found no explicit stable, alpha or beta versions.
I found no scientific article on Vim 
(this might be the first one), although there are
books, software and third-party documentation on the web.

\section{Basics}\label{basics}
Vim's interface is text-based.
In the GUI mode (gVim),
there are convenient menus and toolbars
but all functionalities are still available though
the command line mode.
Vimscript is the internal language of Vim,
and is often used for scripting by users
although other languages might be used 
(e.g. Python, Perl, Lua, Racker, Ruby and Tcl). 
Each line of a Vimscript is a command on the
command-line mode.
This section can be thought as a tutorial
that focuses on the namespaces, i.e. sets of tokens
that carry values or triggers procedures.

\subsection{The bare minimum}\label{minimum}
You open a file with Vim by executing
the command: \texttt{vim <filename>}.
Inside Vim, you start in the normal
mode, and might want to move around using
\texttt{h-j-k-l} for left-down-up-right.
To insert characters, move your
cursor to the desired location an press i,
which puts Vim in the insert mode.
Go back to normal mode by pressing
\texttt{<ESC>} or \texttt{<C-C>}.
You save the file by typing \texttt{:w<CR>},
and exit Vim by typing \texttt{:q<CR>}.
You can save and quit with \tttt{:wq<CR>}
or \tttt{:x<CR>} or \tttt{ZZ}.

\subsection{Vim help}
You can find help for vim in various places.
The standard resource is the Vim help files.
They are accessed by typing \texttt{:h <anything><CR>}
in normal mode.
Examples of such \texttt{<anything>} are:
color, navigation, :vs, vimtutor.
Type 
\texttt{:h usr\_toc<CR>}
to access the official User Manual,
which is considerably lengthy and complex
and is usually not read by users for a few years.
In learning Vim, one
might want to run the \texttt{vimtutor} command
(outside Vim) to start the Vim Tutor.

There are good resources on the Web for learning
and tweaking Vim:
\begin{itemize}
  \item ``Vim Adventures'' is an online RPG game for practicing
  and memorizing Vim commands while having fun.
  This game is quite famous among Vim users.
  \item There are official and semi-official Vim sites e.g.:
  \url{www.vim.org}, \url{https://www.vi-improved.org} and
  \url{http://vim.wikia.com/}.
  \item Many hacks, understandings and general issues
  (e.g. how to make such a move) are asked and answered
  in online platforms (e.g. Quora, Stack Overflow, Stack Exchange, Reddit, Email list, IRC Channel).
  One often finds these links through a search engine.
  \item You can find many videos about Vim.
  One traditional site is \url{http://vimcasts.org},
  but you might find them by a search engine (e.g. \url{http://derekwyatt.org/vim/tutorials/}) or in Youtube and Vimeo.
\end{itemize}

% any other commands? files?
\subsection{Namespaces}\label{namespaces}
Vim is a text editor ouroboros~\cite{ouroWiki} because
text and writing alters text and writing.
You have namespaces where tokens have scalar
or complex values.

\subsubsection{Commands and mappings}
There are commands, typing sequences which trigger automated actions,
for each mode:
\begin{itemize}
	\item in Normal mode all keys are mapped to commands.
		There is redundancy and additional commands
		using Ctrl and Shift keys.
		Some keys expect a second key,
		and have available possibilities,
		specially the z and g.
    See Section~\ref{normal} and Appendix~\ref{notes}
    for more insights into the commands available in the
    normal mode.
	\item In the other modes, the sequences available for mappings are more obvious and abundant.
		One should look at \ttt{:h index} to know about all the standard mappings
		and use \tttt{:map} to list the user-defined mappings.
\end{itemize}

\ttt{C-\textbackslash} is always reserved for extensions,
which makes it a safe namespace to use (while there are no
such extensions).

A colon command can be written as a string
and executed by the \tttt{:execute} colon command.
E.g. \tttt{:execute 'vs afile.txt'}.
As there are colon commands that execute commands in other
modes, e.g. \tttt{:normal ?\^def },
the \tttt{:execute} is a way to build commands in any mode,
e.g.
\tttt{:execute 'normal' tabn 'gt'}.


\subsubsection{Variables}
There are some types of variables in Vim:
\begin{itemize}
	\item Environment variables: names start with \$ an hold system
		variables, such as \tttt{\$PATH} and \tttt{\$PWD}.
	\item Option variables: names start with a \& and are meant to control the behavior of the editor.
		One might change a value through set or let, e.g.
		\tttt{:set bg=light} or \tttt{:let \&bg=light}.
	\item Registers: start with @ and are meant for automation and transfer of texts (copy and paste).
	\item Internal variables are created with let and preferably have a prefix:
	\tttt{b:}, \tttt{w:}, \tttt{t:}, \tttt{l:}, \tttt{s:},
		are local to the buffer, window, tab page, function, and
		sourced Vim file, respectively.
 \tttt{v:}, \tttt{g:} are global, the first are predefined by Vim.
		\tttt{a:} is for function arguments.
		If there is no prefix, the variable is global or internal to a function if occurring inside a function.
		More about internal variables in \tttt{:h internal-variables}.
	\item The value of a variable can be a scalar, string, list, dictionary, function reference, etc (see \tttt{:h eval}).
\end{itemize}
You can echo any of such variables or use in expressions.
Notice that you will only be able to echo a \tttt{b:} variable inside
the buffer where it is defined.
For all the Vimscript capabilities, including loops, conditionals,
and builting functions, refer to \ttt{:h vim-script}.
Classes are possible only in rudimentary forms, e.g. through dictionaries,
but the language is otherwise overall quite powerful,
specially in dealing with text and editor behavior, as expected


\subsubsection{State lists}\label{state}
Vim keeps a number of lists which expresses the state of the editor.
These are examples of useful lists:
\begin{itemize}
	\item The entered commands are accessed through \tttt{:hist a}. 
    The tokens a / s : can be used for specific types of commands, such
		as search and colon commands.
	\item File buffers are kept with numeric ids. See buffers with \tttt{:ls} and load a buffer to the window with \tttt{:b <num or token in file name>}.
  \item The windows open are listed in \tttt{:ls} with a character \tttt{a} in the second column, and are listed with \tttt{:tabs}.
	\item Tabs list can be reached through \tttt{:tabs}.
		It is usual both to show and hide the tabs bar (mapping in~\cite{vimrc}).
	\item Jumps are available through \tttt{:jumps}.
		One positions the cursor at each jump through \ttt{C-o} and \ttt{C-i}.
  \item Registers are available through \tttt{:reg} command, as variables and through shortcuts in different modes.
    They also keep track or your copy, edition and deletion and are promptly defined
    by recording a typing sequence with the \tttt{q} normal command.
	\item An undo list can be accessed with \tttt{:changes}.
	\item A list with all the sourced scripts in a Vim instance is displayed through \tttt{:scriptnames}.
	\item The markers defined are listed with the \tttt{:marks} command.
		These are set by \tttt{mX} in normal mode, where X is the marker identifier.
		Uppercase letters are cross buffer.
	\item Quickfix and Location lists which are populated through \tttt{:vim} and \tttt{:make}
		and variations, such as \tttt{:grep}.
		One might run \tttt{:vim /section/ \%} and then \tttt{:copen}
		to open the Quickfix window, where the lines of occurrence are in sequence
		and one can \ttt{CR} one of them to have the cursor in the main window active at
		the first character of the match.
		One might run \tttt{:lvim /section/ \%} and then \tttt{:lopen}
		to use the location-list window instead of the Quickfix, which
		is very similar, but one per window instead of one per buffer.
		More information in \tttt{:h quickfix}.
	\item A tags list have to be made so one can use tags.
		Most often one will generate the tags list using the exuberant ctags,
		which supports dozens of languages.
		E.g. \tttt{:!ctags-exuberant functions.py} or \tttt{!ctags-exuberant -R ./},
		and then using \ttt{C-]} to
		go to the position of tag under cursor, and \tttt{:tabe tags} to open
		the tags file.
	\item The argument list holds a list of files to be edited or browsed. 
		The list can be input at Vim startup (e.g. \tttt{\$ vim file1.txt file2.py})
	or using commands (e.g. \tttt{:ar ./*}).
		The file being edited is changed by \tttt{:n} and \tttt{:p} commands,
		one might perform actions on each file in argument list using \tttt{:argdo}.
		All files in argument list is also in the buffers list.
    For further information, see \tttt{:h arglist} and Section~\ref{netrw}.
\end{itemize}

A file with information about the state of the editor
can be achieved through: \tttt{:source \$VIMRUNTIME/bugreport.vim},
and this script might be consulted because it has
a collection of commands to access various settings of Vim.
Another good list of commands to know about Vim's state is kept on
\url{http://vim.wikia.com/wiki/Displaying_the_current_Vim_environment}.
I would specially highlight the \tttt{:syntax} command because
it displays the tokens and related type of meanings when run
inside e.g. a \tttt{.vim} or help file.

\subsubsection{On the persistence of visual cues about the editor state}\label{visual1}

You can keep track of the editor state though commands and persistent visual
cues, specially the tabs bar, the status line, and the line
reserved for the command-line.
For state persistence, one might keep an undo file for each file as
in~\cite{vimrc}.
Sessions are easy to manage, enabling one to save and load the
editor's state, with the opened windows, tabs, buffers, etc.
I use the mappings in~\cite{vimrc} because they keep the sessions
in a reasonable directory and makes it easier to remember and tweak
then the standard commands to deal with sessions.
More information in \tttt{:h sessions}.
One might use \tttt{:h views} to keep the state of one window,
but sessions keep all the states from all windows.
This entails a strategy to deal with Vim that is similar to
the use of Byobu/Tmux/Screen.
The main limitation I found to this approach is that
Vim is not keeping track of the terminals opened.
If you open a terminal inside Vim with the \tttt{:term}
command, you will save the session as usual, but when loading
you get dummy empty windows for them and an error message.
Screen and Tmux are the most popular terminal multiplexers.
Byobu, built on top of them, has awesome keyboard shortcuts
for managing sessions, windows and splits of terminals.
Byoby/Screen/Tmux keep the state for future load.

Because browsing the interface in Vim is fast,
and it favors copy and paste of text,
I tend to keeep a tab with some terminals:
one with an IPython shell, another two for compiling latex
and opening PDF files (e.g. with \tttt{\$ evince <filename.pdf>}).
One enters normal mode in terminal with \tttt{C-W N}
and it is very corfortable to copy and browse the bash history of the
terminal.
I found the mappings on~\cite{vimrc} very helpful for directing
editor focus to splits (\ttt{C-hjkl}) and tabs (\tttt{gr}, \tttt{gt}),
which I make available across terminal and normal modes.
But I've been thinking on using \ttt{C-} commands also
for tabs (not only for splits).

A good strategy I find is to have selective visual cues of the state to make persistent or hide. A mapping to toggle each of them.
Now I toggle byobu/screen/tmux bar with \ttt{F5},
status line and tabs bar with \ttt{localleader-T or B} according to script~\cite{vimrc}.
I am mostly using the cleanest setting, toggling on the tabs bar and status line sometimes. Numbering is always there. I rarely turn them off but keep the mappings \ttt{leader-n or N} to toggle just in case.
Instead of keeping the status bar, I use \ttt{C-g} to know about the
file and \ttt{gC-g} to know more and rarily.
It seems not possible to remove the statusbar between horizontal splits.
After asking around and experimenting, I realized that it seems
reasonable to keep at least one line dividing the windows, so
if it comes to it, I just \tttt{set statusline=-}.
Unfortunatelly, as far as I could dig, one will need to enter
Vim code to enable a horizontal split without losing a line.
My ideal of this feature would be to have a visual cue of the
first and/or last line of the windows in the lines-number
column, or complete the spaces and empty chars with \$\$\$\$ or so.

Autocommands are the standard way to define event-trigered routines in Vim.
These are often related to particular file types, but are also often
in defining the automated behavior of Vim.
In~\cite{vimrc} is found an autocommand for keeping track of the last inserted texts
in the \tttt{".lkjh} registers (\tttt{@.lkjh} variables).

\subsection{Using Vim's modes}\label{modes}
Vim has some basic and fully implemented modes of usage:
\begin{itemize}
  \item Normal mode: used for changing
  the position of the cursor or the text displayed
  at the window.
  A core goal of the normal mode it to allow fast
  navigation of the document while allowing
  the typist to maintain its fingers on the home row
  (i.e. on the center of the keyboard).
  The mode is also used for manupulating text
  (e.g. copy, paste, delete, change case) and
    changing to other modes.
  \item Insert mode: for inserting text.
  \item Command-line mode: for entering Ex commands.
  \item Ex mode: similar to command-line mode,
  but more specialized for running various Ex commands.
  \item Visual mode: for making, manipulating and navigating
  selections of texts.
  \item Select mode: similar to visual mode but
  favors CUA\footnote{IBM Common User Access: \url{https://en.wikipedia.org/wiki/IBM_Common_User_Access}.}.
\end{itemize}

There is another basic mode, but it is not fully implemented:
the Terminal-Job mode.
There are seven additional modes which are mostly subordinate 
to the basic modes and will be described when convenient.
The manual page for Vim modes can be accessed by typing
\texttt{:h vim-mode}.
Some of the modes are now further considered for
an overview of the Vim usage possibilites.

\subsubsection{Normal mode}\label{normal}
Sometimes also called navigation or command mode,
the normal mode is most powerful for
navigating, manipulating texts and changing to other modes.

The simplest of these three is changing to other modes:
type any of these letters to change to insert mode:
\texttt{iIaAoOsScC}. More on the transition between
normal and insert mode on Section~\ref{navIn}.
Type any of these characters to change to command-line mode:
\texttt{:/?}.
Type \texttt{Q} to enter Ex mode.
Type \texttt{v}, \texttt{V}, \texttt{CTRL+V} to enter visual mode.

For most basic and naive navigation, one should check Section~\ref{minimum}.
Most often, one uses:
\begin{itemize}
  \item \texttt{Ctrl+(d,u,f,b)} for half-page down and up
and whole page down and up, although these commands might
be set to scroll a different number of lines.
  \item \texttt{Ctrl+(e,y)} to move the window one line down or up.
  \item \texttt{(w,b,e)} to move to the next, previous and end-of-next word. There motions iterate over sequences of characters separated
    by special characters (e.g. punctuation and parenthesis) as
    specified by the output of \texttt{:se iskeyword}.
    To iterate over space-separated tokens, use \texttt{W,B,E}).
    To move to the end of last word, one might use \texttt{be}
    or \texttt{ge}.
The \texttt{),(} commands iterate through sentences,
		\tttt{\},\{} through text blocks separated by empty lines.
  \item \texttt{(fX,tX,FX, TX)} to move to or just before any X character,
	  \tttt{;} and \tttt{,} for next and previous found character.
  \item Search with \tttt{/} or \tttt{?}, although these are in truth command-line commands.
  \item \texttt{CTRL+(o,i)} to move to an older or newer position in
    jump list.
  \item \texttt{'X,`X} to move the cursor to a mark bond to the alphanumeric character X: \texttt{`X} moves to the exact position while \texttt{'X} moves to
    the first non-blank character of the line.
    A mark is registered by the user in any cursor position
    by typing \texttt{mX}, where X is any letter.
    If X is lowercase, the mark is local to the buffer (the file),
    if it is uppercase or numeric, it is global to the Vim session.
\end{itemize}

There are many more facilities to navigate Vim
as explained in \texttt{:h navigation}.

For changing the text, usual commands include:
\begin{itemize}
  \item \texttt{d\{motion\}} to delete the characters
    involved in the motion command.
  \item \texttt{dd,D} to delete a line or from the cursor to the end of the line.
  \item \texttt{x,X} to delete the character under or before the cursor.
  \item \texttt{$\sim$} to swap the case of a character.
  \item \texttt{gu\{motion\},gU\{motion\},g$\sim$\{motion\}} to make lowercase, uppercase or switch the case of the characters involved in the motion.
\end{itemize}

There are way more commands to change the texts.
Some of them are discussed in Section~\ref{navIn}
because they involve a transition to the insertion mode.
A thorough consideration of the commands in the normal mode
is found by executing \texttt{:h navigation} and 
\texttt{:h change.txt}.

\subsubsection{Insertion}
Once in the insertion mode, the character keys
input the characters at the cursor position at the current buffer.
One can exit insert mode by pressing \texttt{<Esc>},
and Vim will be put in normal mode.
Most useful commands in insert mode include:
\begin{itemize}
  \item \texttt{Ctrl+o} to execute one and only command in normal mode.
	  This enter a secondary mode (see Section~\ref{modes})
  \item \ttt{C-R} to paste a register (a variable starting with
    '@', defined, copied or recorded through a \tttt{q} command in
    normal mode and as covered in Section~\ref{state}).
  \item \ttt{C-T} to indent current line.
  \item \ttt{C-U,W} to delete all chars from cursor to the beginning of
    the line, the previous word.
  \item \ttt{C-N,P} to find next, pervious keywords that match the prefix at hand..
  \item \ttt{C-X} commands for scrolling the window with multiple
    \ttt{C-E} and \ttt{C-Y} strokes and for some completion facilities.
\end{itemize}

\subsubsection{Command-line mode}
This mode is dedicated for writing colon, search and filter commands,
entered through typing \tttt{:}, \tttt{?}, \tttt{/} and \tttt{!} in normal or visual modes.
Most useful commands in this mode include:
\begin{itemize}
  \item \ttt{C-B} and \ttt{C-E} to move cursor to the beginning and
    end of the line.
  \item \ttt{C-W} and \ttt{C-U} to delete last word or everything
    until the cursor.
  \item \ttt{C-R} to paste a register as in insert mode.
\end{itemize}

\subsubsection{Terminal-Job mode}\label{terminal}
This mode is reported to not have reached a stable usage design
(see \tttt{:h terminal}.
I find that it works exceptionaly well and have used it to run
scripts in an IPython shell, compile latex files, and even open PDFs and images.
Vim browsing of windows and text manipulation is well developed,
so the terminal mode enables a very convenient integration,
more traditionally achieved through Byobu/Tmux/Screen terminal
multiplexers.
Most useful commands in terminal-job mode include:
\begin{itemize}
  \item \tttt{<C-W>\_N} and \tttt{<C-W>\_:} for entering the normal
    and command-line modes.
  \item \tttt{<C-W>\_"} to paste a register.
  \item \tttt{i} for entering the terminal-job mode from the normal mode.
\end{itemize}

It is useful to define the same mappings for nativating splits and tabs
for both normal and terminal-job modes, as in~\cite{vimrc}.

\subsubsection{Normal $\rightarrow$ insertion}\label{navIn}
Many commands bridge from Normal to Insertion modes:
\tttt{csrCSR}, \tttt{iws}, etc.
These make convenient the replacement of text and populates registers.
The absence of a short command to insert one token is
a known issue in Vim.
Reasonable mappings to insert and append a token to and around
other tokens are in~\cite{vimrc}.
Vim commands are designed to couple operator and motion commands.
There are many operator commands that take the editor from
the normal mode to the insert mode, most of them
favoring deletion or change, as detailed in \tttt{:h operator}.
Motion commands are described in \tttt{:h motion}.

\subsubsection{Netrw}\label{netrw}
The standard interface of Vim for browsing file trees in Netrw.
It starts when you open a directory, such as with \tttt{:e .<CR>}.
It has solid support for editing remote files (such as over ssh or
ftp) and handy e.g. mappings to open the files as splits and tabs (specially \tttt{pot}).
Most useful commands in netrw include:
\begin{itemize}
  \item \tttt{d} and \tttt{\%} for creating directories and files.
    \ttt{Delete} removes both.
  \item \tttt{mf} and \tttt{mb} for marking files and bookmarking directories.
  \item \tttt{gb} and \tttt{uU} are used to load directories
    while marked files might be copied, moved, edited, greped, tagged and migrated
    to and from the argslist as in \tttt{:h netrw-mf}.
\end{itemize}
Last notes:
there is no insert mode in netrw interface;
the commands in \tttt{:h netrw-explore} are
handy for opening the directory of the
file being edited;
further information in \tttt{:h netrw}.

\subsubsection{Ex}
This might be the least popular mode.
One might use \tttt{q:,/,?} to have a window
with colon or search command history to be edited normaly,
and the chosen command can be run with \ttt{CR}.
Vimscript was largerly based on the ex editor~\cite{ex},
and a more advanced user might use it for prototyping
by defining mappings and settings and managing sessions and scripts
propably in a \tttt{plugin/} folder of a directory in \tttt{:echo \&runtimepath}.
In the default interface started with the \tttt{Q} command in normal mode,
each command is input without entering : again. Use \tttt{:vi} to exit Ex mode,
follow documentation from \tttt{:h Ex-mode} for further information.

\subsection{Standard configuration files and directories and my .vim/vimrc}
You can check the scripts Vim loads by using the debug script mentioned
in Section~\ref{state}.
By default, \tttt{~/.vimrc} and \tttt{~/.vim/vimrc} files are run by Vim at the beginning of the startup.
One might edit the vimrc file with \tttt{:e \$MYVIMRC}
and reload it with \tttt{:source \$MYVIMRC}, but
I use the mappings in~\cite{vimrc} to encourage the constant enhancement of my settings.
Any other file might be included to run at startup by
adding a line \tttt{:source <afile.vim>} in vimrc.
In fact, it is on vimrc that one usually specifies the plugins
and plugin managers they use.
Use an \tttt{after/} folder of a directory in \tttt{:se runtimepath}
or follow some patterns described on the next section to change the
scripts and sequence of them to be loaded.

\subsection{Plugins and packages}
One can see the list of standard plugins
with \tttt{:h standard-plugin-list} command.
Any \tttt{plugins/**/*.vim} file inside a directory
listed by \tttt{:se runtimepath} will be loaded
(e.g. \tttt{.vim/plugin/something/ascript.vim}).
There are various ways to automate the installation
and enhnace the management of plugins.
By default, one has the GETSCRIPT interface (see \tttt{:h getscript}),
that downloads latest scripts from sourceforge as specified in \tttt{:h getscript-data},
and the Vimball interface, which creates and loads a Vimball for a plugin
with \tttt{:[range]MkVimball <filename> path}, where range specifies lines
that hold paths to files to be included in the \tttt{<filename>.vba} Vimball.
The Vimball can be installed in a system by \tttt{:source <filename>.vba}
or loading it at vim startup with \tttt{\$ vim <filename>.vba}.

A Vim package is a directory that contain plugins.
It should be located inside a \tttt{pack/} directory
somewhere in the directories listed with \tttt{:se runtimepath}.
The plugins found in \tttt{pack/<packName>/*/start/} are loaded
at startup, the plugins found at
\tttt{pack/<packName>/opt/**} are loaded with\\
\tttt{:packdd <script\_or\_directory\_name>}.

All directories vim looks for scripts are described
in \tttt{:h vimfiles} and are basically set by
\tttt{:se runtimepath} and conventions inside
each path therein, such as to look for \tttt{vimrc} files
and \tttt{plugin/} or \tttt{pack/} directories.
Vim scripts can be loaded conditionaly,
e.g. only if a function is used as in \tttt{:h autoload-functions}.
Example of such are filetype plugins (enabled by in a \tttt{ftplugin/<filetype>.vim} file,
e.g. inside a plugin directory).
There is a number of plugin managers for Vim.
Pathogen and Vundle seem to be the most popular,
one because of its minimalism, the other because of
advanced features, e.g. for searching and installing plugins with
colon commands.

\subsection{Spell and spelllang (en and pt\_br)}
One might set the spelling language with\\ \tttt{:se spelllang=en\_us}
or \tttt{:se spelllang=pt\_br}
and toggle spell checking with \tttt{:setl spell!}.
These are used so often that one might use shortcuts as in~\cite{vimrc}.
Currently, Vim will download the files for a specific language if
not found in system.

\subsection{Scripting, Functions, Vimscript and other languages (e.g.  Python)}
In Vimscript,
the colon commands (also Ex or command-line commands) are related through spaces,
punctuations and keywords (see \texttt{:h script}).
Scripting the Vim editor can also be accomplished using other languages,
as well documented e.g. in \tttt{:h python}.
Functions are defined through colon commands and are called
inside colon commands e.g. \tttt{:call MyFunction()}
or \tttt{:echo MyFunction(4)}.
Notice that functions are not commands but might be binded to them
through colon commands e.g. \tttt{nnoremap gF :call MyFunction()<CR>}.
Executing source files is very straightforward with \tttt{:so
<filename>.vim},
and one can always use the Ex mode for rapid scripting.

At the same time, the \tttt{:term} and terminal-job mode make
scripting other software more convenient, as output is
proptly navigated and copied. The shell works well for generic use,
such as opening images and PDFs through \tttt{\$ eog} and \tttt{\$ evince}.
See Section~\ref{terminal} for further directions about the terminal.

\subsection{Color}\label{visual2}
Newer versions of Vim support 24 bit true color (aka 16 million colors)
in terminal Vim (not only in GVim).
The terminal must support true color, and tests are available e.g. in
\url{https://gist.github.com/XVilka/8346728}.
Then Vim needs to be set to use true colors with
\tttt{:set termguicolors}.
If using 8 or 16 bit colors, Vim uses the color pallete from the
terminal, if using true colors each color is defined directly.
Settings for using true colors inside Byobu/Tmux involve tweakings
and are available in~\cite{vimrc}.
Good color schemes to use with true colors are Gruvbox
and Solarize.
One might source syntax files at any time to change syntax, usually
though linking tokens to syntax groups with
\tttt{:syntax keyword <group\_name> <token1> <token2> <token3> ...}
and then relating the group to another group 
\tttt{:highlight link <group\_name> <group\_name2>}
or by specifying the group characteristics directly:
\tttt{highlight <group\_name> guifg=\#ffffff}.
If you change a syntax file, reloading a file with with \tttt{:e<CR>}
updates the highlighting on the window with the corresponding filetype.
A complete syntax highlighting support typically involves at least three files:
\begin{itemize}
  \item \tttt{ftdetect/<filetype>.vim}, where the file type is
    detected with e.g.
    \tttt{:autocmd BufNewFile,BufRead *.<file\_extension> setfiletype <filetype>}
  \item \tttt{ftplugin/<filetype>.vim} with general settings for the
    file type, such as: 
    \tttt{:set tabstop=2 softtabstop=2 shiftwidth=2 textwidth=70
    expandtab autoindent}.
  \item A \tttt{syntax/<filetype>.vim} file, with bindings between
    tokens and highlighting groups; and highlighting group definitions.
\end{itemize}

This very scheme is implemented very straightforward in this
plugin~\cite{tokipona}
for highlighting text in the Toki-Pona language~\cite{tpLang}.
Syntax highlighting plugins are as filetype plugins,
but also have a \tttt{syntax/<filetype>.vim} file
relating keywords to highlighting groups. 
One might see every highlighting group, and their final visual results,
with the command
\tttt{:so \$VIMRUNTIME/syntax/hitest.vim}.
In my system, the \tttt{ftdetect/} and \tttt{ftplugin/} folders load as expected in the \tttt{plugin/}
directory, but \tttt{syntax/} files had to be moved to \tttt{$\sim$/.vim/syntax/}.

\subsection{Fonts}
The fonts are defined by your terminal or inside GVim.
\ttt{C-+} and \ttt{C--} can be used to change font size.
Some settings for fonts, such as boldface, might be set using
the syntax highlighting facilities described in the previous section.

\section{Conclusions and further work}\label{conc}
This document seems reasonable as an overall reference of the Vim editor,
at least for my usage and proficiency.
Given the folkloric milestone of using Vim for 10 years,
this article might serve as a benchmark for one to relate
it's current use and understandings.
As a pedagogical material, it seems to be unique in the emphasis
on namespaces, understood as commands, variables, state-related lists, etc,
especially in Section~\ref{namespaces},
and the reference to the standard Vim documentation
to achieve the DRY KISS style described in Section~\ref{intro}.

\noindent {\bf Potential enhancements} to this document are:
\begin{itemize}
  \item The inclusion of facilities such as reading emails and connecting over ssh.
    There is a working hack in~\cite{vimrc} for browsing over the WWW,
    but such aspect of Vim usage might receive more attention
    given that it is confortable to use the navigation and editing facilities
    and the resulting integrated environment.
  \item Updating of the information I can find about the issues discussed,
    such as about status lines in Section~\ref{visual1}.
  \item Include a discussion about Neovim.
    I have never used it, but it seems to be reaching a considerable user base
    and it might be feasable to give an account of Vim and Neovim
    after some tests and researching the official and user base documentation.
  \item Better cite documentation, plugins and Vim authors.
    I prefered to keep the references inline through \tttt{:h} commands
    and URLs, more in accordance with the style of Vim documentation,
    but bibliographical items constitute a valuable asset for academic literature,
    and some authors might find their work more pretiged if more
    thoroughly cited.
\end{itemize}

\noindent {\bf Potential next steps} in using Vim:
\begin{itemize}
  \item Measure the performance of text mining routines in Vim against those implemented in C or Python.
  \item Enhance the HTTP browsing capabilities of~\cite{vimrc}.
  \item Better integrate Python and Vimscript, especially for data visualization
    and syntax highlighting management.
    In accordance with the visualization issues described in
    Sections~\ref{visual1} and~\ref{visual2}.
  \item Make plugins for:
    \begin{itemize}
      \item listing all the mappings available in each mode and the typing combinations which are available
        for new mappings.
      \item Sessions, as described in Section~\ref{visual1}.
      \item AA messages (shouts): to keep track, document and share of working sessions
        as in~\cite{aa1,aa2}, with capabilities to manage AA sessions,
        send visual or sonic cues for temporal marks, use Vim state to build AA shouts,
        relate AA sessions to other media, such as software repositories,
        screencasts and images, interact with IRC channels and other social platforms.
      \item Slick Vim: a collection of the settings, mappings and plugins I use.
        Enhancements such as using \ttt{C-} commands also to browse tabs,
        shortcuts to join a window into a tab,
        and dummy minimal plugins as simplest, file type and syntax highlighting,
        Some more elaborate tweaks should also be present, such
        as breaking lines in sensible places for natural language texts
        while respecting e.g. \tttt{:se textwidth}.
      \item Dealing with .swo and .swp temporary files.
        In summary, if the restored .swo file has the same content
        and the file being opened,
        the restoration phase can be omitted.
        If the contents differ, Vim should open a tab with each file
        in a vertical split and run \tttt{:windo :diffthis}.
      \item Rendering images and equations.
        These are useful for using Vim in presentations
        or achieving a textual representation when it is mandatory,
        such as to comply with the limitations of a platform (e.g. Vim editor). 
        but also hold stylistic merits as ascii art is often
        very appreciated.
        One can both obtain an ascii representation of a binary image (e.g. JPG, PNG),
        and can directly render ascii charts from data using cues such are shape, position
        and color.
      \item Redirecting the commands that usually show the results in a 'more' interface,
        which cannot be searched nor copied,
        to a parsed and linked quickfix or location window.
        The basic idea is to use \tttt{:redir} command to redirect the output
        of such commands with \tttt{:se nomore}.
      \item Slide presentations. I've been using some automation for browsing
        slides and opening figures and some Vim users asked for the settings and
        commands. They are very elementary use of registers being executed over
        arbitrary but consistent textual patterns.
      \item Changing the color scheme an highlighting scheme incrementally
        and selectively using the features described in Section~\ref{visual2}.
    \end{itemize}
\end{itemize}

\subsection*{Acknowledgments}
FAPESP (project 2017/05838-3); Vim developers and documentation maintainers;
Vim user community. 

\appendix
\section{Example of usage session}
I usually begin by opening a file or directory
with \tttt{\$ vim <filename>}.
The color scheme is alternated between
blue and GruvBox with
\tttt{:colo gruvbox} and \tttt{:colo blue}.
I open an vertical split and then move
the window to a new tab using
\tttt{:vs} and \ttt{C-W\_T}.
I then search tokens related to
the enhancements I want to make or
the knowledge I want to acquire.
I go back to the previous tab with \tttt{gr}
and make a global replace with
\tttt{:\%s/<this>/<that/g}.
On adding dots to sentences,
I record in the \tttt{"q} register
the sequence \tttt{jA.jj},
using it as a macro 10 times by
\tttt{10@q}.
I move to the other tab with \tttt{gt}
and open a terminal window with \tttt{:term}
for compiling latex files.
I start another terminal with \tttt{<C-W> :term}
for opening the resulting PDFs with evince.
If any new idea comes to mind and I have time,
\tttt{\textbackslash\textbackslash s} opens 
my vimrc~\cite{vimrc} for editing and \tttt{\textbackslash s}
sources it.
If there is e.g. code or notes in other projects I am working
on, I reach them through \tttt{<Alt-arrows>} as other Vim instances
in Byobu/Tmux sessions and windows.

\section{My vimrc file and usage}
In this Appendix is my vimrc file.
Although it has comments, one should
look for the options that (s)he does
not understand, as a thorough explanation
of the file is tedious and out of the scope
of this document.

In using Vim with my vimrc file~\cite{vimrc},
I mostly toggle the status line with \texttt{\textbackslash\textbackslash B}
and the tab line with \texttt{\textbackslash\textbackslash T}.
Save and close windows with \texttt{\textbackslash w} and \texttt{\textbackslash q}.
The mappings for transitioning through splits and tabs
are also used constantly.

\section{Example of notes on mappings}\label{notes}
\texttt{:h index} shows all the default mappings
while \tttt{:map} shows the user-defined mappings.
By considering such information, one can make
useful observations exemplified in this Appendix.

\subsection{Normal mode}
Every letter and character in the keyboard is used.
In Normal mode: \ttt{TAB} is the same as \ttt{C-I},
\ttt{BS} and \ttt{C-H} is the same as h.
\ttt{C-J} and \ttt{C-N} is the same as j.
\ttt{C-P} is the same as k.
Space is same as l.
\ttt{C-[} and \ttt{Esc} are not used
\tttt{<C-\textbackslash> a-z} is reserved for extensions
\ttt{C-\_} not used.
\tttt{+} is the same as \ttt{CR} and they are both not very useful.
Del is same as x.

Many \tttt{][} combinations are not used, e.g.
with \tttt{abhjklfg}.
The \tttt{\_} command might be used as a more powerful \tttt{\^},
leaving it free for mappings.

Directions, home, end, page up and down, insert, all have mappings
in more centraly located keys.
unbounded: g,z,[]

The \ttt{C-(HJKL)} commands are redundant,
with the exception of \ttt{C-L} which redraws the screen,
so it is a reasonable choice to use them to move focus
of the editor to splits in the \tttt{hjkl} directions.

Many key combinations are available for new mappings through the \tttt{g}
and \tttt{z}, and \tttt{v} commands.
They have typical uses, e.g. \tttt{z} for folds,
spell checking and some movements (mainly when wrap is set).

\subsection{Insert mode}
All standard char keys are used for entering chars on text.
\tttt{<C-G>(j,k)} can be achieved by \tttt{<C-O>(j,k)} which is more powerful
in moving through multiple lines.
\ttt{C-[} is the same as \ttt{ESC}.
\ttt{C-J} and \ttt{C-M} are \ttt{CR}.
\tttt{<C-\textbackslash> a-z} is reserved for extensions,
other combinations with \ttt{C-\textbackslash} are not used.

\section{My \tttt{:version}}
I should keep the output of \tttt{:version} executed on the system
in which I wrote this document in this link:
\url{}.
It was compiled with this Makefile~\cite{makefile}
in a 16.04 LTS Ubuntu Linux,
and is tagged as 8.0, Included Patches 1-1173.

\begin{thebibliography}{99}
\fontsize{11}{0}\selectfont
\bibitem{anPh}
	Anthropological physics and social psychology in the critical research of networks. Complex Networks Digital Campus (CS-DC'15).
	Available at \url{https://youtu.be/oeOKYc3-nbM}
\bibitem{anPh2}
	Fabbri, R. What are you and I? [anthropological physics fundamentals], 2015. Available at \url{https://www.academia.edu/10356773/What_are_you_and_I_anthropological_physics_fundamentals_}
\bibitem{ouroWiki}
  Ouroboros. (2017, November 2). In Wikipedia, The Free Encyclopedia. Retrieved 22:19, November 9, 2017, from \url{https://en.wikipedia.org/w/index.php?title=Ouroboros&oldid=808392809}
\bibitem{vimrc}
	Fabbri, R. (2017). A reasonable vimrc file. Available at \url{https://raw.githubusercontent.com/ttm/vim/master/vimrc} 
\bibitem{ex}
  Ex (text editor). (2017, March 22). In Wikipedia, The Free Encyclopedia. Retrieved 22:22, November 9, 2017, from \url{https://en.wikipedia.org/w/index.php?title=Ex_(text_editor)&oldid=771621020}
\bibitem{tokipona}
	Fabbri, R. (2017). A Toki Pona Python Package and Vim Syntax Highlighting. Available at \url{https://github.com/ttm/tokipona} 
\bibitem{tpLang}
	Lang, S. (2014). Toki Pona: the language of good. Tawhid Publishing.
    ISBN-10: 0978292308, ISBN-13: 978-0978292300.
\bibitem{aa1}
	Fabbri, R., Fabbri, R., Vieira, V., Penalva, D., Shiga, D., Mendonça, M., Negrao, A., Zambianchi, L., \& Thumé, G. (2013). AA: The Algorithmic Autoregulation (Distributed Software Development) Methodology. RESI. From \url{https://arxiv.org/abs/1604.08255}
\bibitem{aa2}
	Fabbri, R. (2017).
The Algorithmic-Autoregulation (AA) Methodology and Software:
a collective focus on self-transparency. ENMC2017. From \url{https://github.com/ttm/ensaaio/raw/master/emc/article.pdf} 
\bibitem{makefile}
	Fabbri, R. (2017). The Makefile with which I compiled Vim for this article. Available at \url{https://raw.githubusercontent.com/ttm/vim/master/Makefile} 
\end{thebibliography}
\end{document}
