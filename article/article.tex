\documentclass{article}
\usepackage{graphicx}
\usepackage[usenames,dvipsnames]{xcolor}
\usepackage{hyperref}
\hypersetup{
	%pagebackref=true,
	pdfcreator={LaTeX with abnTeX2},
	pdfkeywords={abnt}{latex}{abntex}{USPSC}{trabalho acadêmico}, 
	colorlinks=true,       		% false: boxed links; true: colored links
	linkcolor=blue,          	% color of internal links
	citecolor=blue,        		% color of links to bibliography
	filecolor=magenta,      		% color of file links
	urlcolor=blue,
	allbordercolors=black,
	bookmarksdepth=4
}
\usepackage[utf8]{inputenc}

\begin{document}
\title{On the Vim editor:\\
vim features and tweaks after 10 years of usage}

\author{Renato Fabbri\\
\texttt{renato.fabbri@gmail.com}\\
University of São Paulo,\\
Institute of Mathematical and Computer Sciences\\
São Carlos, SP, Brazil
}
\maketitle
\begin{abstract}
The Vim editor is very rich in capabilities
and thus complex.
This article is a description of the Vim text editor
and a set of enhancements proposed for it.
It is the result of more than ten years of experience
in using Vim for writing and editing various types of documents,
mostly:
Python, C++, JavaScript, ChucK, etc, programs;
and \LaTeX, Markdown, HTML, RDF, Make and other markup files;
binary files.
It is said that it takes about ten years to master this
text editor, and I find that other experienced users
have a different view of Vim and that they use a different
set of tools.
Therefore, this document exposes my insights in order
to confront my usage with the praxis of the Vim users community
and to make available a reference document with which new users
can grasp a sound overview by reading it and the discussions that
it might generate.
Also, it should be useful for users of any degree of experience,
including me, as a compendium of commands, namespaces and tweaks.
Upon feedback, and maturing of my Vim usage,
this document might be enhanced or receive additional
material.
\end{abstract}

\section{Introduction}
Vim is a very complex editor,
considered by the Linux community to be
matched only by Emacs.
They both are the standard advanced text editors
of the open source community
and have been developed for decades.
This document describes the Vim editor
and proposes a set of enhancements of the user
experience which
are implemented as Vim Plugins.
The editor has a very mature documentation
and the contents herein presented is a
report on the overall understandings I
have of Vim after a little bit more
of ten years using it.
The purposes of this document are:
\begin{itemize}
  \item to help new users in grasping Vim essentials
  and convenient practices.
  \item To make a sound overview of the editor available.
  \item To record the view of the editor that
  an user (me) has after $10$ years of usage.
  \item To confront my usage with that of other experienced
  users. This is helpful for me, but also for the other users
  as they might benefit from this content and from discussions
  that might arise from it.
\end{itemize}

I've had experience with other editors, e.g. Kate, gedit, and Notepad2.
I used Vim for writting and editing computer code (Python, Javascript, C++, ChucK, bash, etc), markup languages (HTML, CSS, RDF, Markdown, \LaTeX, etc) and binary files.
Eventually, I edited database files and other types os files.
With Vim, I mostly write software (web and scientific),
music, poems and short stories.
It is very useful because:
\begin{itemize}
  \item it is meant to be a plain text (e.g. ascii, utf8) editor
  and does not (by standard) insert special charaters (e.g. for formating, with binary instructions).
  \item It has a powerful architecture and set of commands.
  \item It is highly configurable and most often the users
  has a set of commands.
\end{itemize}

The standard capabilities of the editor
should become clear in Section~\ref{basics}.
Vim is constantly evolving and there are many plugins,
some of them very popular for both general and specific
type of users.
Accordingly, there are many possible enhancements,
and Section~\ref{issues} report the most prominent of them
for me and potential workarounds made available as plugins.

% Vi and Vim
\subsection{Historical note}
Vim was first released publicly in 1991.
It is a cross-platform GNU licensed free and open source extended clone of Bill Joy's vi text editor.
Vim's development is coordinated (and performed) since the beginning
mainly by Bram Moolenaar.
Today, current stable version is reported with 8.0.1173
in the commit messages but tagged with 8.0.1257.
I found no explicit alpha or beta versions
and found no scientific article on Vim (this might be the first one).

\section{Basics}\label{basics}
Vim's interface is text-based.
In the GUI mode (gVim),
there are convenient menus and toolbars
but all functionalities are still available though
the command line mode.
Vimscript is the internal language of Vim,
and is often used for scripting by users
although other languages might be used 
(e.g. Python, Perl, Lua, Racker, Ruby and Tcl). 

\subsection{The bare minimum}\label{minimum}
You open a file with Vim by executing
the command: \texttt{vim <filename>}.
Inside Vim, you start in the normal
mode, and might want to move around using
\texttt{h-j-k-l} for left-down-up-right.
To insert characters, move your
cursor to the desired location an press i,
which puts Vim in the insert mode.
Go back to normal mode by pressing
\texttt{<ESC>}.
You save the file by typing \texttt{:w<CR>},
you exit vim by typing \texttt{:q<CR>}.

\subsection{Vim help}
You can find help for vim in various places.
The standard resource is the Vim help resources.
They are accessed by typing \texttt{:h <anything><CR>}
in normal mode.
Examples of such \texttt{<anything>} are:
colour, navigation, :vs, vimtutor.
Type 
\texttt{:h usr\_toc<CR>}
to access the official Vim User Manual,
which is considerably lengthy and complex
and is usually not read by users for a few years.
In learning Vim, one
might want to run the \texttt{vimtutor} command
(outside Vim) to start the Vim Tutor.

There are good resources on the Web for learning
and tweaking Vim:
\begin{itemize}
  \item ``Vim Adventures'' is an online RPG game for practicing
  and memorizing Vim commands while having fun.
  This game is quite famous among Vim users.
  \item There are official and semi-official Vim sites:
  \url{www.vim.org}, \url{https://www.vi-improved.org} and
  \url{http://vim.wikia.com/}.
  \item Many hacks, understandings and general issues
  (e.g. how to make such a move) are asked and answered
  in online forums (e.g. Quora, Stack Overflow, Stack Exchange, Reddit).
  One often finds these links through a search engine.
  \item You can find many videos about Vim.
  One traditional site is \url{http://vimcasts.org},
  but you might find them by a search engine (e.g. \url{http://derekwyatt.org/vim/tutorials/}) or in Youtube and Vimeo.
\end{itemize}

There are mailing lists and IRC channels listed in these
sites for getting help on Vim usage and development.

% any other commands? files?

\subsection{Using Vim's modes}
Vim has some basic modes of usage:
\begin{itemize}
  \item Normal mode: used for changing
  the position of the cursor or the text displayed
  at the window.
  A code goal of the normal mode it to allow fast
  navigation of the document while allowing
  the typist to maintain its fingers on the home row
  (i.e. on the center of the keyboard).
  The mode is also used for manupulating text
  (e.g. copy, paste, delete, change case).
  \item Insert mode: for inserting text.
  \item Command-line mode: for entering Ex commands.
  \item Ex mode: similar to command-line mode,
  but more specialized for running various Ex commands.
  \item Visual mode: for making, manipulating and navigating
  selections of texts.
  \item Select mode: similar to visual mode but
  favors CUA\footnote{IBM Common User Access: \url{https://en.wikipedia.org/wiki/IBM_Common_User_Access}.}.
\end{itemize}

There are seven additional modes which are mostly subordinate 
to the basic modes and will be described when convenient.
The manual page for Vim modes can be accessed by typing
\texttt{:h vim-mode}.

\subsubsection{Normal mode}
Sometimes also called navigation or command mode,
the normal mode is most powerful for
navigating, manipulating texts and changing to other modes.

The simplest of these three is changing to other modes:
type any of these letters to change to insert mode:
\texttt{iIaAoOsScC}. More on the transition between
normal and insert mode on Section~\ref{navIn}.
Type any of these characters to change to command-line mode:
\texttt{:/?}.
Type \texttt{Q} to enter Ex mode.
Type \texttt{v}, \texttt{V}, \texttt{CTRL+V} to enter visual mode.

For most basic and naive navigation, one should check Section~\ref{minimum}.
Most often, one uses:
\begin{itemize}
  \item \texttt{Ctrl+(d,u,f,b)} for half-page down and up
and whole page down and up, although these commands might
be set to scroll a different number of lines.
  \item \texttt{Ctrl+(e,y)} to move the window one line down or up.
  \item \texttt{(w,b,e)} to move to the next, previous and end-of-next word. There motions iterate over sequences of characters separated
    by special characters (e.g. punctuation and parenthesis) as
    specified by the output of \texttt{:se iskeyword}.
    To interate over space-separated tokens, use \texttt{W,B,E}).
    To move to the end of last word, one might use \texttt{be}
    or \texttt{ge}.
  \item \texttt{(fX,tX)} to move to or just before any X character.
  \item \texttt{\},\{} for moving to the next blank line.
  \item Search with / or ?, although these are in truth command-line 
  \item \texttt{CTRL+(o,i)} to move to an older or newer position in
    jump list.
  \item \texttt{'X,`X} to move the cursor to a mark bond to the alphanumeric character X: \texttt{`X} moves to the exact position while \texttt{'X} moves to
    the first non-blank character of the line.
    A mark is registered by the user in any cursor position
    by typing \texttt{mX}, where X is any letter.
    If X is lowercase, the mark is local to the buffer (the file),
    if it is uppercase or numeric, it is global to the Vim session.
\end{itemize}

There are many more facilities to navigate Vim
as explained in \texttt{:h navigation}.
E.g. one might find useful the \texttt{),(} commands
for iterating through sentences.

For changing the text, usual commands include:
\begin{itemize}
  \item \texttt{d\{motion\}} to delete the characters
    involved in the motion command.
  \item \texttt{dd,D} to delete a line or from the cursor to the end of the line.
  \item \texttt{x,X} to delete the character under or before the cursor  \item \texttt{~} to swap the case of a character.
  \item \texttt{gu\{motion\},gU\{motion\},g~\{motion\}} to make lowercase, uppercase or switch the case of the characters involved in the motion.
\end{itemize}

There are way more commands to change the texts.
Some of them are discussed in Section~\ref{navIn}
because they involve a transition to the insertion mode.
A thorough consideration of the commands in the command mode
is found by executing \texttt{:h navigation} and 
\texttt{:h change.txt}.

\subsubsection{Insertion}
Once in the insertion mode, the character keys
input the characters at the cursor position.
One can exit insert mode by pressing \texttt{<Esc>},
and Vim will be put in normal mode.
Most used commands in insert mode include:
\begin{itemize}
  \item \texttt{Ctrl+o} to execute one and only command in normal mode.
  \item \texttt{asdjn}
\end{itemize}


\subsubsection{Command-line mode}
% enter by typing :, ?, / in normal or visual modes.


\subsection{Navigation + insertion}\label{navIn}
Commands that bridge from Navigation to Insertion:
csrCSR, iws, etc
Default, line and block visual selection:
  and AiIcCdD

\subsection{Netrw}
Most important help files on Netrw
No insert mode.

Edit directory
Sexplore
Create and delete files and directories
open files: on window, split, preview
Change display of files
Bookmarks
Most important help files on Netrw


\subsection{Standard configuration files and directories and My .vim/vimrc}
\subsection{Spell and spelllang (en and pt\_br)}
\subsection{Tabs, splits, buffers and namespaces}
\& \% \$ and the following
\subsection{Mappings and abbreviations}
\subsection{Macros and registers for copy and paste}
\subsection{History of commands}
\subsection{Lists of Vim state variables}
% of markers, jumps, registers, commands (hist)
% :
\subsection{Undo}
\subsection{Scripting, Functions, Vimscript and Python}
\subsection{Plugins}
\subsubsection{Standard features}
\subsubsection{Writting plugins}
\subsubsection{Plugin systems: usage for using and writing plugins}
\subsection{Colour}
\subsubsection{Standard, 8 and 16 bits, and true color, Screen/Byobu}
\subsubsection{Gruvebox, Solarize and other colourschemes}
\subsection{Fonts}
\subsubsection{Cools nd popular fonts}
\subsubsection{How to set fonts in xterm, gnome-terminal and GVim.}
\subsection{Verbosing, logs and possibilities of using it to study your own}
\subsection{age (often used commands and typed sequences).}
\subsection{Highlighting}
\subsection{Bash and Vim commands}
\subsection{Compiling, standard features and plugins}
\subsection{Quickfix}
\subsection{Persistence and vimrc}
% markers, undo, buffers, registers, bookmarked directories,
% vimrc for settings
\section{Issues and plugins}\label{issues}

\section{Final words and further work}

Potential enhancements to this document are:
\begin{itemize}
  \item The inclusion of reading emails and connecting over ssh.
\end{itemize}

\appendix
\section{Example of usage session}
% open file, split, tab, find, replace char,
% append to line, visual block, etc.
\section{Key naming in Vim documentation}
% CTRL, <C-X> <C-S-X>
% Special chars..
\section{My vimrc file}
In this Appendix is my vimrc file.
Although it has comments, one should
look for the options that (s)he does
not understand, as a thorough explanation
of the file is tedious and out of the scope
of this document.

\section{Overview of my usage of Vim}
% historical usage, other editors,
% most often edited file types

\end{document}
